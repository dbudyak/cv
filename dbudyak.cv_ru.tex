\documentclass[a4paper,10pt]{article}

\usepackage[utf8]{inputenc}
\usepackage[russian]{babel}
\usepackage[cm-default]{fontspec}

\usepackage{marvosym}
\usepackage{fontspec} 				
\usepackage{xunicode,xltxtra,url,parskip}
\RequirePackage{color,graphicx}
\usepackage[usenames,dvipsnames]{xcolor}
\usepackage[big]{layaureo} 				
\usepackage{supertabular} 				
\usepackage{titlesec}					
\usepackage{hyperref}

\definecolor{linkcolour}{rgb}{0,0.2,0.6}
\hypersetup{colorlinks,breaklinks,urlcolor=linkcolour, linkcolor=linkcolour}

\defaultfontfeatures{Mapping=tex-text}
\setmainfont[SmallCapsFont = Days.otf,BoldFont = EtelkaMed.otf,ItalicFont = Blogger.ttf]{MetaPro.otf}


\titleformat{\section}{\Large\scshape\raggedright}{}{0em}{}[\titlerule]
\titlespacing{\section}{0pt}{3pt}{3pt}

\hyphenation{im-pre-se}

\usepackage[absolute]{textpos}

\setlength{\TPHorizModule}{20mm}
\setlength{\TPVertModule}{\TPHorizModule}
\textblockorigin{2mm}{0.65\paperheight}
\setlength{\parindent}{0pt}

\begin{document}

\pagestyle{empty}
\par{\centering
	\Huge \textsc{Дмитрий Будяк}
	}\bigskip\par

\section{Персональные данные}
	\begin{tabular}{rl}
	\textsc{День рождения:} &  26 Апреля 1990 \\
	\textsc{Моб. телефон:}     & +7 981 8330844 \\
	\textsc{Email:}     & \href{mailto:dmitry.budyak@gmail.com}{dmitry.budyak@gmail.com} \\
	\textsc{LinkedIn:} & \href{http://ru.linkedin.com/in/dmitrybudyak}{http://ru.linkedin.com/in/dmitrybudyak}
	\end{tabular}

\section{Опыт работы}
	\begin{tabular}{r|p{11cm}}
	  \textsc{Сейчас} & \textsc{Oracle Development SPB} \\ & \emph{QA инженер} \\
	  \textsc{Дек 2013}&\textbf{Internet of Things Platform}\\&
	  %\footnotesize{
	  \begin{itemize}
	  	\item Сперва работал в отделе тестирования Java ME. В задачи входили прогоны тестов на различных платформах с помощью JT Harness, настройки тестовых конфигураций, написание мидлетов для работы с сенсорами устройств с помощью Java ME. 
		\item Через полгода присоединился к команде тестирования IoT.
		\item Разрабатывал функциональные тесты для тестирования Gateway-части платформы, также участвуя в разработке тестового фреймворка.
		\item В ходе работы активно использовал возможности Java 8, знакомился с Git и Gradle.
		\item Использовал собыйтийно-ориентированные методы для разработки тестов.
		\item Научился работать с OSGi-контейнерами, такими как Prosyst и Felix.
		\item Писал Test Design Specifications для функциональных тестов.
		\item Проводил ручное тестирование Gateway c помощью bluetooth-устройств, подключенных к платформам Raspberry Pi и IMX Sabrelite.
		\item Писал скрипты на Bash для автоматизирования повседневных задач.
		\item В данный момент продолжаю работать над тестовым фреймворком и разработкой функциональных тестов.
	\end{itemize}
	  %}
	  \\
	    \multicolumn{2}{c}{} \\
		\textsc{Авг 2013}& \textsc{OOO ``Айти++''}\\&\emph{Андроид разработчик} \\
		\textsc{Сен 2012}& Разработка медийных приложений и их интеграция с социальными сервисами.\\
		  \multicolumn{2}{c}{} \\
		\textsc{Июль 2010} & 	\textsc{OOO ``Наш район''}\\&\emph{Программист} \\
		\textsc{Июнь 2010}& Поддержка работы веб-сайта на ajax, jquery, php.\\
		  \multicolumn{2}{c}{} \\
		\textsc{Март 2008} & 	\textsc{OAO ``Сталепрокатный завод''}\\&\emph{Помощник системного администратора}\\
		\textsc{Сент 2007}&Администрирование корпоративной сети, сервера и пользовательских хостов.
		\end{tabular}

\section{Образование}
	\begin{tabular}{rl}	
	  \textsc{Июль 2014} & Магистр \textbf{Прикладной математики и информатики} в \\& \textsc{СПБНИУ ИТМО}\\
			& Тема диплома: ``Компьютерное моделирование квантовых оптических сетей''\\&
			Разработка приложения на JavaFX, позволяющего моделировать квантовые\\& схемы с помощью оптических элементов.
	\end{tabular}

\section{Языки}
	\begin{tabular}{rl}
		\textsc{Русский:}&Родной\\
		\textsc{Английский:}&Pre-Intermediate\\
	\end{tabular}

\section{Навыки}
	\begin{tabular}{rl}
		Билд-системы: ANT, Gradle \\
		Системы контроля версий: SVN, Git \\
		Языки: Java SE 6/7/8, Haskell \\
		Скриптовые языки: bash \\ 
		Инструменты: R, Octave \\
		Тестирование: TestNG, Selenium \\
		Рабочее окружение: Gentoo Linux \\
		Также: JavaFX, MS SQL, Java ME, Android SDK
	\end{tabular}

\section{Интересы и увлечения}
	ИТ, технологии, свободное ПО, Gentoo, Java, математика, квантовые компьютеры,
	психология, походы, sci-fi, роликовые коньки. 
\end{document}
