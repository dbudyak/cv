\documentclass[a4paper,10pt]{article}

\usepackage[utf8]{inputenc}
\usepackage[russian]{babel}
\usepackage[cm-default]{fontspec}

\usepackage{marvosym}
\usepackage{fontspec} 				
\usepackage{xunicode,xltxtra,url,parskip}
\RequirePackage{color,graphicx}
\usepackage[usenames,dvipsnames]{xcolor}
\usepackage[big]{layaureo} 				
\usepackage{supertabular} 				
\usepackage{titlesec}					
\usepackage{hyperref}

\definecolor{linkcolour}{rgb}{0,0.2,0.6}
\hypersetup{colorlinks,breaklinks,urlcolor=linkcolour, linkcolor=linkcolour}

\defaultfontfeatures{Mapping=tex-text}
\setmainfont[SmallCapsFont = Days.otf,BoldFont = EtelkaMed.otf,ItalicFont = Blogger.ttf]{MetaPro.otf}


\titleformat{\section}{\large\scshape\raggedright}{}{0em}{}[\titlerule]
\titlespacing{\section}{0pt}{3pt}{3pt}

\hyphenation{im-pre-se}

\usepackage[absolute]{textpos}

\setlength{\TPHorizModule}{20mm}
\setlength{\TPVertModule}{\TPHorizModule}
\textblockorigin{2mm}{0.65\paperheight}
\setlength{\parindent}{0pt}

\begin{document}
\par{\centering
	\Large \textsc{Будяк Дмитрий Романович}
	}
\newline
\section{Контакты}
	\begin{tabular}{rl}
	\textsc{Моб. телефон:}     & +7 981 8330844 \\
	\textsc{Email:}     & \href{mailto:dmitry.budyak@gmail.com}{dmitry.budyak@gmail.com} \\
	\textsc{LinkedIn:} & \href{http://ru.linkedin.com/in/dmitrybudyak}{http://ru.linkedin.com/in/dmitrybudyak}
	\end{tabular}

\section{Опыт работы}
	\begin{tabular}{l|p{12cm}}
		\multicolumn{2}{c}{} \\
			\textsc{сейчас}  & 
				\textsc{Фриланс}\\ &
				\emph{Android разработчик} \\
			\textsc{март 2016}  & Разработка социального приложения, базирующегося на XMPP\\
		\multicolumn{2}{c}{} \\
			\textsc{март 2016} & 
				\textsc{OOO ``ПМБК''}\\ &
				\emph{инженер-программист} \\
			\textsc{май 2015} &
					\begin{itemize}
					\item Участие в разработке модулей парсинга и хранения данных от поставщиков спортивных мероприятий
					\item Использование Java 8, JPA/EclipseLink, PlayFramework 2.1					
					\end{itemize} \\
		\multicolumn{2}{c}{} \\
	  		\textsc{май 2015} & 
				\textsc{Oracle Development SPB} \\ & 
				\emph{QA инженер} \\
	  		\textsc{дек 2013} &
	  			\begin{itemize}
	  			\item Работа в отделе тестирования Java ME. Прогоны тестов на различных платформах с помощью JT Harness, настройка тестовых конфигураций, написание мидлетов для работы с сенсорами устройств с помощью Java ME; спустя полгода присоединился к команде тестирования Internet of Things Platform.
				\item Разработка функциональных тестов для тестирования Gateway-части платформы, участие в разработке тестового фреймворка.
				\item Написание Test Design Specifications для функциональных тестов.
				\item Ручное тестирование Gateway-части платформы c помощью bluetooth-устройств, подключенных к платформам Raspberry Pi и IMX Sabrelite.
				\item Автоматизирование повседневных задач на bash.
				\end{itemize} \\
	    	\multicolumn{2}{c}{} \\
			\textsc{авг 2013}  & 
				\textsc{OOO ``Айти++''}\\ &
				\emph{Android разработчик} \\
			\textsc{сен 2012}  & Разработка медийных приложений и их интеграция с социальными сервисами.\\
		\multicolumn{2}{c}{} \\
			\textsc{июль 2010} & 
				\textsc{OOO ``Наш район''}\\ & 	
				\emph{программист} \\
			\textsc{июнь 2010} & Поддержка работы веб-сайта на ajax, jquery, php.\\
		\multicolumn{2}{c}{} \\
			\textsc{март 2008} & 
				\textsc{OAO ``Сталепрокатный завод''}\\&
				\emph{помощник системного администратора}\\
			\textsc{сент 2007} & Администрирование корпоративной сети, сервера и пользовательских хостов.
	\end{tabular}
\newpage
\section{Образование}
	\begin{tabular}{rl}	
	  \textsc{Июль 2014} & Магистр \textbf{Прикладной математики и информатики} в \\& \textsc{СПБНИУ ИТМО}\\
			& Тема диплома: ``Компьютерное моделирование квантовых оптических сетей''\\&
			Разработка приложения на JavaFX, позволяющего моделировать оптические квантовые схемы.
	\end{tabular}
\newline
\section{Языки}
	\begin{tabular}{ll}
	\multicolumn{2}{c}{} \\
		Английский: & Pre-Intermediate 
	\end{tabular}
\newline
\section{Навыки}
	\begin{tabular}{ll}
	\multicolumn{2}{c}{} \\
		Билд-системы: & gradle, maven, ant \\
		Системы контроля версий: & git, mercurial, svn  \\
		Языки: & Java \\
		Скриптовые языки: & bash \\
		Тестирование: & TestNG, Selenium \\
		Рабочее окружение: & Gentoo Linux \\
		Также: & JavaFX, Java ME, Android SDK
	\end{tabular}
\newline
\section{Интересы и увлечения}
	ИТ, технологии, свободное ПО, Gentoo, Java, математика,
	психология, походы, sci-fi, роликовые коньки. 
\end{document}
